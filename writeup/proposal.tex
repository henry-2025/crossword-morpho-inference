% This must be in the first 5 lines to tell arXiv to use pdfLaTeX, which is strongly recommended.
\pdfoutput=1
% In particular, the hyperref package requires pdfLaTeX in order to break URLs across lines.

\documentclass[11pt]{article}

% Remove the "review" option to generate the final version.
\usepackage{acl}

% Standard package includes
\usepackage{times}
\usepackage{latexsym}

% For proper rendering and hyphenation of words containing Latin characters (including in bib files)
\usepackage[T1]{fontenc}
% For Vietnamese characters
% \usepackage[T5]{fontenc}
% See https://www.latex-project.org/help/documentation/encguide.pdf for other character sets

% This assumes your files are encoded as UTF8
\usepackage[utf8]{inputenc}

% This is not strictly necessary, and may be commented out,
% but it will improve the layout of the manuscript,
% and will typically save some space.
\usepackage{microtype}

% If the title and author information does not fit in the area allocated, uncomment the following
%
%\setlength\titlebox{<dim>}
%
% and set <dim> to something 5cm or larger.

\title{Final Project Pitch}

% Author information can be set in various styles:
% For several authors from the same institution:
% \author{Author 1 \and ... \and Author n \\
%         Address line \\ ... \\ Address line}
% if the names do not fit well on one line use
%         Author 1 \\ {\bf Author 2} \\ ... \\ {\bf Author n} \\
% For authors from different institutions:
% \author{Author 1 \\ Address line \\  ... \\ Address line
%         \And  ... \And
%         Author n \\ Address line \\ ... \\ Address line}
% To start a seperate ``row'' of authors use \AND, as in
% \author{Author 1 \\ Address line \\  ... \\ Address line
%         \AND
%         Author 2 \\ Address line \\ ... \\ Address line \And
%         Author 3 \\ Address line \\ ... \\ Address line}

\author{Miles Christensen \\
  Harvey Mudd College \\
  \texttt{mchristensen@g.hmc.edu} \\\And
  Henry Pick\\
  Harvey Mudd College\\
  \texttt{hpick@hmc.edu}}
  

\begin{document}
\maketitle

\section{Project Pitch}
As an avid crossword puzzle enthusiast, solving crosswords programmatically presents a really interesting challenge. However, writing a full crossword solver is likely too difficult a task, so instead I'd like to see if I can create a model that will accurately tag a crossword clue with its corresponding part of speech. This can be very useful, as clues and answers are consistent in puzzles -- if a clue is a past tense verb or a plural noun, the answer will be a past tense verb or a plural noun, respectively. I plan to use a data set of 6,000,000 crossword clues found at \url{https://xd.saul.pw/}. The relevant data here will be the clue and only the part of speech of the answer, not necessarily the answer itself; I can use a prebuilt POS tagger to process the answers from this data set, then split the set into training, dev, and test sets as necessary. Using a Naive Bayes classifier, I can train a model that will give a probability distribution for each possible part of speech. I can evaluate this with the test data in two ways: looking simply at the highest probability part of speech and determining whether or not the model was correct, or looking at the probability the model assigned to the correct part of speech.

\section{Related Work}

The task of searching based on representation of meaning, or semantic search, is a well-developed field in NLP and has applications in numerous search systems whether on the Web or within closed systems to generate relevant results. This is a far more advanced topic of information retrieval, usually involving more than just PoS prediction. However, our topic is the same general task of information retrieval based on the features of a query. 

\section{Methods}
\end{document}